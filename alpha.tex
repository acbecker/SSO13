As described in Section~\ref{sec:}, our analysis requires a measure of
the orbital separation relative to the stellar radius, $\alpha = a /
R_{s}$, to derive $e_{min}$ (Eq.\ 7).  The separation is obtained
trivially from Kepler's 3rd Law according to, $a =
(M_{s}*P_{orb}^{2})^{2/3}$, given the mass of the host star,
$\rm{M_{s}}$, and the orbital period of the planet, $P$.  The orbital
period is well known for each planet \citep{Batalha2012}.  However,
the mass of an isolated star is not directly observable.  Instead, it
must be inferred by comparing observed stellar properties to
theoretical stellar evolution models and/or empirical calibrations.

Thus, to derive the stellar mass, we will interpolate state-of-the-art
stellar evolution models from the Dartmouth group \citep{Dotter2008}
in three parameters: effective temperature ($T_{eff}$), metallicity
([Fe/H]), and gravity ($log~g$), as we (L.~Hebb) have done for many
other confirmed transiting planets with radial velocity
measurements \citep[i.e.][]{Hebb2009,Hebb2010,Bouchy2010,Yilen2013}.

Although the stellar parameters are available for all the KOI objects
in the Kepler Input Catalogue (KIC).  These values are based on broad
band colors and are known to have some systematic
problems \citep{Muirhead2012,Pinsonneault2012}.  Therefore, we will
derive the host star mass using the $T_{eff}$ and [Fe/H] derived
directly from spectra through stellar characterization techniques.
While $T_{eff}$ and [Fe/H] can be determined with high precision from
the stellar spectrum $log~g$ is usually poorly constrained, and thus
stellar masses derived from the spectroscopic $log~g$ can have large
uncertainties and can suffer from systematics.

To avoid this problem, we will obtain the $log g$ values based on a
novel characterization of the high frequency variability in the {\it
Kepler} light curves (Bastien, Stassun et al.\ 2013, to appear in
Nature) which has been shown to reproduce the exquisite
astroseismically measured $log g$ values \citep{Huber2013} for dwarf
and subgiant stars.  The authors present a technique that can be used
to derive $log g$ values for the majority of Kepler targets (not just
the bright astroseismology objects) with uncertainties of $\le 0.1$ by
empirically detecting granulation on the stellar surface.  This allows
for significantly more accurate and precise gravity measurements than
can be derived from typical modeling of broad absorption line wings.

We plan to use the spectroscopically and photometrically derived
parameters that are available in the literature to derive the host
star mass for our short period transiting planet sample.  Several
teams have large, observing programs to obtain spectra of this
important sample.  For example, \citet{Everett2013} derived stellar
parameters for $\sim 400$ faint KOIs from low resolution data;
approximately 1000 targets have been observed with Keck HIRES
(J.~Johnson et al.\ in prep) and are currently being analyzed with a
new Spectroscopy Made Easy (SME) pipeline (Cargile, Hebb et al.\ in
prep); and the 2.5m Nordic Optical Telescope (NOT) has has an ongoing
program to derive stellar parameters of Kepler KOIs (L. Buchave, PI).
However, if certain targets lack observed spectra or derived stellar
parameters, we will obtain the necessary data with the ARC 3.5m
echelle spectrograph through the University of Washington's guaranteed
access to the Apache Point Telescopes.  We will analyze these data as
necessary with our SME pipeline, as we have done for many other
targets \citep[i.e.][]{Wisniewski2012}.

We will interpolate the Dartmouth models considering the 1$\sigma$
errors in the photometrically measured $log g$ and in the
spectroscopically determined [Fe/H] and $T_{eff}$.  Typical
uncertainties of $\le 0.1$~dex is [Fe/H], $\le 100$~K in $T_{eff}$,
and $\le 0.1$ in $log g$ result in uncertainties on the resulting
stellar mass of $5-8$\% for well understood F, G and K-dwarf stars.
This error budget includes the formal errors from the measured
uncertainties on the parameters and systematic uncertainties arising
from variation between different stellar evolution models
(2-4\%, \citep{Southworth2009}).  For each target, we will combine the
extremely precise orbital period measured by the {\it Kepler} team and
our derived mass to determine the orbital separation, $a$.  Combined
with the accurate $log~g$ measured from the photometric variability
(as described above), we will produce a catalogue of $\alpha =
a/R_{s}$ values for all the short period KOI transiting planet
candidates in our sample with conservative uncertainties of $\le
10$\%.  Finally, our comparison of the stellar parameters to the
evolutionary models allows us to estimate the age of the planetary
system.

