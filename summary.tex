\centerline{\bf This project summary will go in NSPIRES.  Put here for proofing} \medskip

\centerline{\bf mini--Neptune vs. super--Earth: Orbital Eccentricity as a Proxy for Internal Structure} \medskip

We propose here a Bayesian reanalysis of short--period exoplanet
transits in data from the Kepler spacecraft, in order to constrain the
exoplanet tidal quality factor Q.  The quality factor Q is expected to
be a function of planetary radius, and represents the transition
between gaseous (high Q) and dry, rocky (low Q) bodies.  These two
classes of exoplanets exhibit significant differences in how they
deform due to tides when in short period orbits, leading to
differences in tidal circularization timescales.  We use dynamical
simulations to show that short period (2--10 days) gaseous bodies with
high Q are found with values of eccentricity close to the primordial
distribution, while rocky bodies having low Q are preferentially found
at low eccentricity due to tidal circularization.  Thus a study of the
minimum eccentricities of short--period planets can constrain the
radius of this transition.  This is fundamental to understanding if
confirmed Kepler exoplanets are likely to be uninhabitable
mini--Neptunes, or potentially habitable super--Earths.

The minimum eccentricity is constrained using the difference between
the modeled transit duration, and the transit duration that would be
seen if the orbit were circular.  To assess this difference, we
analyze Kepler lightcurves using a purely geometric model that
includes no assumptions about the orbital dynamics.  We cast our
measurement of minimum eccentricity in terms of two model parameters,
whose posterior distributions we explore using Markov Chain Monte
Carlo methods, and one physical parameter that must be estimated from
other means.  We have run a suite of simulations using a grid in
transit depth, stellar brightness (lightcurve signal--to--noise), and
the number of transits included in the model.  We find {\bf XXX}.  We
apply this modeling code to confirmed exoplanet system Kepler 62--b
and find that we are able to reproduce these results using real data.
We propose here to extend this analysis to an ensemble of {\bf XXX}
KOIs that have been selected based upon their Kepler--reported periods
and planetary radius, and which have supporting information on the
host star density and age.  This reanalysis will enable the first
measurement of Q as a function of planetary radius.

This project spans the fields of high--performance computation,
statistical modeling of experimental data, and theoretical
interpretation of the results, which will make it a valuable
contribution to the field of exoplanet studies.  We will release code
and data using open--source collaboration tools, and help to guide the
adoption of reproducible research standards by releasing analysis
packages as part of our publication process.

