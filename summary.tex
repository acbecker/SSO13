{\bf This project summary will go in NSPIRES.  Put here for proofing.} \medskip 

\centerline{\bf EXOQ} \medskip 

We propose here a Bayesian reanalysis of Kepler exoplanet lightcurves
to constrain the tidal quality factor Q of short period planets via
their minimum eccentricity.  The quality factor Q is expected to be a
function of planetary radius, and represents the transition between
gaseous (high Q) and dry, rocky (low Q) bodies.  These two classes of
exoplanets exhibit significant differences in how they deform due to
tides when in short period orbits, leading to differences in tidal
circularization timescales.  We use dynamical simulations to show that
short period (less than 6 days) gaseous bodies with high Q are found
with values of eccentricity close to the primordial distribution,
while rocky bodies having low Q are preferentially found at low
eccentricity due to tidal circularization.  Thus a study of the
minimum eccentricities of short--period planets can help constrain the
radius of this transition.

The minimum eccentricity is measured using the difference between the
modeled transit duration, and the transit duration that would be seen
if the orbit was circular.  To assess this difference, we analyze
Kepler lightcurves using a purely geometric model that includes no
assumptions about the orbital dynamics.  We cast our measurement of
$e_{min}$ in terms of two model parameters, whose posterior
distributions we explore using Markov Chain Monte Carlo methods, and
one physical parameter which must be estimated from other means.
